\documentclass{book}
\usepackage{cuisine}

\index{}

\begin{document}
    \begin{recipe}[pumpkin_crostata]{Pumpkin crostata} {8 Portions} {Preparation: 1h}

        \freeform Doses for 1 crostata of around 24 cm or for 1 kg ca. of biscuits.
        \ingredient[125]{g}{Butter}
        \ingredient[1]{M}{Egg}
        \ingredient[250]{g}{Flour}
        \ingredient[100]{g}{Sugar}
        \ingredient[1]{pinch}{Salt}
        \ingredient[4]{g}{Baking powder}
        \ingredient[\fr{1}{4}]{}{Lemon zest}
        Keep the butter out of the fridge for 20 minutes at environment temperature.\\
        Cut it in small pieces and put it in a bowl together with the grated lemon zest.\\
        Add the sugar in the bowl and start to mix everything together using your hands.\\
        Add the egg and mix them to the other ingredients using a whip.\\
        Sift the flour together with the baking powder and the salt and spill it on the countertop giving it a fountain shape.\\
        Pour the content of the bowl in the flour and start mixing everything together with the hands,
        (\textit{it's very important to be as quick as possible and to handle the dough for a very short time}).\\
        Cover it with food-quality film and let it rest in the fridge for at least 1 hour.


    \end{recipe}

    \begin{recipe} [crema_catalana] {Crema catalana} {}{}
        \ingredient[500]{ml}{Latte intero}
        \ingredient[\fr{1}{2}]{buccia}{Limone}
        \ingredient[1]{pezzo}{Cannella}
        \ingredient[1]{pezzo}{Vaniglia (stecca)}
        \ingredient[20]{gr}{Frumina (o maizena)}
        \ingredient[4]{tuorli}{}
        \ingredient[80]{gr}{Zucchero}
        \ingredient[qb]{-}{Noce moscata}
        \ingredient[qb]{-}{Zucchero di canna}
        Inserire nel \textbf{boccale} il latte, la scorza del limone, cannella e vaniglia in stecca e cuocere \textbf{10 min. 90º inverso mestolo}.\\\\
        Filtrare il latte con un colino e versarlo in una ciotola.\\\\
        Inserire nel \textbf{boccale} lo zucchero, la frumina e i tuorli e mescolare qualche secondo a \textbf{vel. 3}.\\\\
        Abbassare la velocità a \textbf{2} e iniziare ad aggiungere il latte dal foro nel coperchio.\\\\
        Cuocere la crema\\ \textbf{7 min. - vel. 4 - 90º}.\\
        Versare la crema in coppette monodose e porle in frigorigero.\\\\
        Appena prima di servire ricoprire la crema con uno strato di zucchero di canna e fate dorare lo zucchero con l'aiuto di un cannello.\\



    \end{recipe}

\end{document}